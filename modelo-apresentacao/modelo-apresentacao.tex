%  ________________________________________
%
%   EcompTeX - Modelo Apresentação, Feb '16
%  ________________________________________
%  
%  - Escola Politécnica de Pernambuco
%  -- Por Thiago Ribeiro -- 		
%
%  - Fique à vontade para modificar o documento e deixe ao 
%  seu gosto. Bom proveito! :)
%  
\documentclass[brazil]{beamer}
\let\Tiny=\tiny
%\usepackage{lmodern}
\usefonttheme{serif}
\usepackage[utf8]{inputenc}
\usepackage{graphicx}
\usepackage[brazil]{babel} 
\usepackage{amssymb,amsmath}
\usetheme{Amsterdam}

\begin{document}
	\begin{frame}
		\begin{figure}[!htp]
			{\includegraphics[scale=0.2, width=1.2cm]{../imagens/poli.png}}\hfill%
			{\includegraphics[scale=0.3, width=1.8cm]{../imagens/ecomp.png}}%
		\end{figure}
		\begin{center}
			\small{\textbf{Título da Apresentação}}\\[0.2cm]
			\small{Sub-título da apresentação}\\[0.7cm]
			\begin{center}
				\begin{tabular}{rl}
					\scriptsize{{\bf Disciplina:}} & \scriptsize{{\sf COMP01 - Nome da Disciplina}}\\
					\scriptsize{{\bf Professor:}} & \scriptsize{{\sf Nome do professor}}\\
					\scriptsize{{\bf Aluno:}} & \scriptsize{{\sf Nome do aluno}}\\
					\scriptsize{{\bf Outro:}} & \scriptsize{{\sf Conteúdo adicional}}\\
				\end{tabular}
				\vspace{0.2cm}
				\begin{center}
					\scriptsize{{\sc mês ano}}\\\scriptsize{{\sc Outra informação}}\\
				\end{center}
			\end{center}
		\end{center}
	\end{frame}

	\begin{frame}
		\frametitle{Agenda}
		\tableofcontents[hideallsubsections]
	\end{frame}

	% ============= Introdução =================

	\section{Introdução}

	\subsection{O Problema}

	\begin{frame}
		\frametitle{Introdução}
		\framesubtitle{O Problema}
		%\framesubtitle{O Problema da Poluição Sonora}
		% ============= Conteúdo =================
		\begin{center}
			Quem nunca passou pela seguinte situação?
		\end{center}
	\end{frame}

	\begin{frame}
		\frametitle{Introdução}
		\framesubtitle{Poluição Sonora}
		%\framesubtitle{O Problema da Poluição Sonora}
		% ============= Conteúdo =================
		De fato, em um típico ambiente urbano, rodeado de construções, avenidas, bares e restaurantes, estamos expostos à:\\[0.5cm] 
		\begin{center}
			\textbf{Poluição Sonora}.
		\end{center}
		\pause
		\begin{flushleft}
			Uma das piores consequências:
		\end{flushleft} 
		\begin{center}
			\textit{Perda Auditiva Induzida por Ruído} (PAIR).
		\end{center}
		
	\end{frame}

	\begin{frame}
		\frametitle{Introdução}
		\framesubtitle{Combate à Poluição Sonora}
		%\framesubtitle{Como combater?}
		% ============= Conteúdo =================
		\vspace{0.5cm}
		No contexto da nossa cidade, a Prefeitura da Cidade do Recife (PCR) e outros órgãos ligados, realizam a fiscalização.
	\end{frame}

	\subsection{Motivação do Trabalho}

	\begin{frame}
		\frametitle{Introdução}
		\framesubtitle{Motivação do Trabalho}
		% ============= Conteúdo =================
		A ideia do sistema medidor desenvolvido neste trabalho tem por motivos principais:\\[0.2cm]
		\begin{enumerate}
			\item Diminuir níveis de poluição; %de que forma?
			\item Identificar e punir emissores; %
			\item Monitorar locais remotamente;
			\item Viabilizar dados à população;
		\end{enumerate}
	\end{frame}

	% ===========  Fundamentacao  ================

	\section{Fundamentação Teórica}

	\subsection{O Som e suas características}

	\begin{frame}
		\frametitle{Fundamentação Teórica}
		\framesubtitle{O Som e suas características}
		% ============= Conteúdo =================
		\begin{block}{Definição}
			\textit{Sons} são vibrações se propagando através de um meio, onde um objeto vibrador comprime e espalha moléculas (do meio).
		\end{block}
	\end{frame}

	\begin{frame}
		\frametitle{Fundamentação Teórica}
		\framesubtitle{O Som e suas características}
		% ============= Conteúdo =================
		O ouvido humano consegue perceber variações de pressão e a unidade utilizada para expressar estas medidas é o \textit{Pascal} (Pa). 
		\begin{align}
		   1 \ pascal \ & = 1 \ Pa \ = 1 \ N/m^2
		\end{align}
		\vspace{0.5cm}Homenagem ao cientista e filósofo francês Blaise Pascal.
	\end{frame}	

	\begin{frame}
		\frametitle{Fundamentação Teórica}
		\framesubtitle{O Som e suas características}
		% ============= Conteúdo =================
		O som \textit{mais suave} que podemos escutar tem uma pressão medida de:
		\begin{center}
			$0,000020 \ Pa = 20 \ \mu Pa$
		\end{center}
		Este é o \textbf{limiar da audição}.
	\end{frame}	

	\begin{frame}
		\frametitle{Fundamentação Teórica}
		\framesubtitle{O Som e suas características}
		% ============= Conteúdo =================
		Já um som que nos \textit{causa dor} tem uma pressão medida de:
		\begin{center}
			$2.000.000.000 \ \mu Pa$
		\end{center}
		Este é o \textbf{Limiar da Dor}.
	\end{frame}	

	\begin{frame}
		\frametitle{Fundamentação Teórica}
		\framesubtitle{O Som e suas características}
		% ============= Conteúdo =================
		%Agora imagine no nosso cotidiano, se as medições de som fossem em Pascal?
		\begin{quote}
			Seu estabelecimento está emitindo $200.000 \ \mu Pa$!
		\end{quote}
		ou...
		\begin{quote}
			Dá pra falar mais baixo? a conversa de vocês está medindo $20.000 \ \mu Pa$!
		\end{quote}
		É um tanto quanto inviável.
	\end{frame}	

	\subsection{A Escala Decibél}

	\begin{frame}
		\frametitle{Fundamentação Teórica}
		\framesubtitle{A Escala Decibél}
		% ============= Conteúdo =================
		Uma escala logarítmica resolve o problema dos valores grandes.
	\end{frame}

	\begin{frame}
		\frametitle{Fundamentação Teórica}
		\framesubtitle{Valores em Pa $\times$ Valores em Decibéis}
		% ============= Conteúdo =================
	\end{frame}

	\begin{frame}
		\frametitle{Fundamentação Teórica}
		\framesubtitle{{\it Sound Pressure Level} (SPL)}
		% ============= Conteúdo =================
		Portanto, o valor de \textit{Nível de Pressão Sonora} em decibéis, é obtido pela fórmula:

		\begin{align}\label{eq-dec}
		  SPL \ = \ 20\cdot log_{10} \left(\frac{p}{p_0}\right)
		\end{align}

		sendo $p$ o valor medido em $Pa$, e $p_0$ o valor de referência padrão que é de $20 \ \mu Pa$ -- limiar da audição humana. 
	\end{frame}

	\subsection{Como medir Som}

	\begin{frame}
		\frametitle{Fundamentação Teórica}
		\framesubtitle{Medidor de Níveis de Pressão Sonora (MNPS)}
		% ============= Conteúdo =================
		\vspace{0.4cm}
		\footnotesize{O instrumento utilizado para medir níveis de pressão sonora é chamado \textit{Medidor de Nível de Pressão Sonora} (MNPS). }
	\end{frame}

	\begin{frame}
		\frametitle{Fundamentação Teórica}
		\framesubtitle{Medidor de Níveis de Pressão Sonora (MNPS)}
		% ============= Conteúdo =================
		Algumas das principais desvantagem de um MNPS comum são:

		\begin{itemize}
			\item Valor relativamente alto;
			\item Impossibilidade de medição remota;
			\item Não salva dados medidos;
		\end{itemize}

		A solução mais utilizada é chamada de MNPS Integrador, com custo médio de mais de R\$2000,00 (!).
	\end{frame}

	% ===========  Proposta do Sistema  =============

	\section{Proposta de Sistema}

	\subsection{Projeto do {\it Hardware}}

	\begin{frame}
		\frametitle{Proposta do Sistema}
		% ============= Conteúdo =================
		\begin{block}{Solução Completa}
			Desenvolvimento de um sistema de {\it hardware} e {\it software}, contendo um medidor de baixo custo com capacidade de monitoramento remoto via \textit{software}.
		\end{block}
	\end{frame}

	\begin{frame}
		\frametitle{Proposta do Sistema}
		\framesubtitle{Diagrama do {\it hardware}}
		% ============= Conteúdo =================
		\begin{center}
			\begin{block}{\small{Solução em {\it hardware}}}
				\small{Componentes conectados em placa de protótição, medindo valores de amplitude sonora e enviando para servidor {\it web}.}
			\end{block}
		\end{center}
	\end{frame}

	\begin{frame}
		\frametitle{Proposta do Sistema}
		\framesubtitle{Arduino}
		% ============= Conteúdo =================
	  	\footnotesize{{\bf Propósito no sistema:} atuar como o ``cérebro'' do sistema de {\it hardware}, controlando os demais componentes a partir do programa gravado no seu microcontrolador.}
	\end{frame}

	\begin{frame}
		\frametitle{Proposta do Sistema}
		\framesubtitle{Detector de Som}
		% ============= Conteúdo =================
	  	{\bf Propósito no sistema:} leitura da amplitude sonora do ambiente.
	\end{frame}

	\begin{frame}
		\frametitle{Proposta do Sistema}
		\framesubtitle{Módulo ESP8266}
		% ============= Conteúdo =================
	  	{\bf Propósito no sistema:} equipar o sistema com Internet ao conectar-se ao WiFi do local.
	\end{frame}

	\begin{frame}
		\frametitle{Proposta do Sistema}
		\framesubtitle{{\it Liquid Crystal Display} (LCD)}
		% ============= Conteúdo =================
	  	{\bf Propósito no sistema:} dispositivo de saída para que o usuário possa verificar valores medidos.
	\end{frame}

	\begin{frame}
		\frametitle{Proposta do Sistema}
		\framesubtitle{{\it Light Emitting Diode} (LED)}
		% ============= Conteúdo =================
	  	{\bf Propósito no sistema:} indicar se os níveis emitidos pelo local estão adequados ou não, de maneira intuitiva.
	\end{frame}

	\subsection{Projeto do {\it Software}}

	\begin{frame}
		\frametitle{Proposta do Sistema}
		\framesubtitle{Desenvolvimento do \textit{Software}}
		% ============= Conteúdo =================
		\vspace{0.5cm}
		\begin{block}{\small{Solução em {\it software}}}
				\small{Aplicação em {\it software} para iOS, permitindo leitura dos valores enviados pelo medidor em {\it hardware}.}
			\end{block}
	\end{frame}

	% ===========  Resultados  =============

	\section{Resultados e Testes}

	\subsection{Medição no {\it Hardware}}

	\begin{frame}
		\frametitle{Resultados e Testes}
		\framesubtitle{Medição no {\it Hardware}}
		% ============= Conteúdo =================
		\vspace{0.5cm}
		Os componentes do medidor foram montados em uma placa de prototipação MSB-500, sendo testados na placa tanto com alimentação de uma bateria de 9V quanto USB (5V).

		\vspace{1cm}
		
		A calibração do valor de referência foi feita a partir do aplicativo {\it Decibels}.
	\end{frame}

	\begin{frame}
		\frametitle{Resultados e Testes}
		\framesubtitle{Medição no {\it Hardware}}
		% ============= Conteúdo =================
	\end{frame}

	\begin{frame}
		\frametitle{Resultados e Testes}
		\framesubtitle{Medição no {\it Hardware}}
		% ============= Conteúdo =================
		\vspace{0.2cm}
		\footnotesize{Permanecendo ligado por volta de 1 hora e 40 minutos numa sala com pessoas conversando, os valores medidos e enviados para o servidor {\it web} podem ser vistos no gráfico.}
	\end{frame}

	\subsection{Medição no {\it Software}}

	\begin{frame}
		\frametitle{Resultados e Testes}
		\framesubtitle{{\it Software} desenvolvido}
		% ============= Conteúdo =================
		\footnotesize{A aplicação foi desenvolvida, compilada e testada em um aparelho iPhone 5, conectado à Internet. Foi possível abrir o {\it app} e ler o último valor calculado e enviados ao servidor através do {\it hardware} desenvolvido.}

	\end{frame}

	\begin{frame}
		\frametitle{Resultados e Testes}
		\framesubtitle{{\it Software} desenvolvido}
		% ============= Conteúdo =================
		\footnotesize{$\blacklozenge$ Pontos no mapa do Centro do Recife: locais que podem ser monitorados.}
	\end{frame}

	\begin{frame}
		\frametitle{Resultados e Testes}
		\framesubtitle{{\it Software} desenvolvido}
		% ============= Conteúdo =================
		\begin{center}
			\footnotesize{$\blacklozenge$ Usuário pode escolher que pontos quer saber valores medidos.}
		\end{center}
	\end{frame}

		\begin{frame}
		\frametitle{Resultados e Testes}
		\framesubtitle{{\it Software} desenvolvido}
		% ============= Conteúdo =================
		\begin{center}
			\footnotesize{$\blacklozenge$ Usuário abre tela de Detalhes e vê último valor medido.}
		\end{center}
	\end{frame}

	% ===========  Trabalhos Futuros  =============

	\section{Trabalhos Futuros}

	\subsection{Novas ideias e melhoramentos}

	\begin{frame}
		\frametitle{Trabalhos Futuros}
		\framesubtitle{Novas ideias e melhoramentos}
		% ============= Conteúdo =================
		\footnotesize{Uma das principais ideias futuras é projetar uma placa de circuito impresso que abrigue todos os componentes. Porém também pretende-se:}
		\begin{itemize}
			\item Desenvolver aplicação multi-plataforma em {\it software}: {\it Android} e {\it Windows Phone};
			\item Melhorar a precisão a partir da calibração com MNPS profissional;
			\item Implementar {\it features} de envio de {\it e-mail} e ligação do {\it app} desenvolvido;
		\end{itemize}
	\end{frame}

\end{document}