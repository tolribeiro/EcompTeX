%  _____________________________________
%
%   Trabalho Acadêmico (LaTeX template)
%  _____________________________________
%  
%  * Universidade de Pernambuco 
%  * Por Thiago Ribeiro. 		
%
%  Fique à vontade para modificar o documento e deixe ao 
%  seu gosto. Bom proveito! :)

\documentclass[12pt]{article}
\usepackage[misc]{ifsym}
% Pacotes para expressões matemáticas
\usepackage{amsmath}
\usepackage{MnSymbol}
\usepackage{wasysym}
% Ajuste de margens
\usepackage[top=1.5cm, left=3cm, right=3cm, bottom=2cm, headheight=110pt]{geometry} 
\usepackage[brazilian]{babel}
\usepackage[utf8]{inputenc}\usepackage[T1]{fontenc}
\usepackage{graphicx}
\usepackage{color, colortbl}
% Header
\usepackage{fancyhdr}
\usepackage{framed}
\pagestyle{fancy}
\usepackage{cite}

\linespread{1.27} % espaçamento entre linhas 

% Início do documento
\begin{document}
	
	\thispagestyle{empty}

	\begin{center}
		\begin{figure}[!htp]
		    {\includegraphics[scale=0.2, width=2.4cm]{../imagens/poli.png}}\hfill%
		    {\includegraphics[scale=0.3, width=3.7cm]{../imagens/ecomp.png}}%
  		\end{figure}
  		\begin{framed}
  			Escola Politécnica de Pernambuco (POLI - UPE) \\ Engenharia de Computação\\[0.2cm] \textsf{CCMP0039} \\ \textsf{Teoria da Computação}
  		\end{framed} 
	\end{center}
	
	\begin{flushleft}
		\textbf{Primeiro Exercício Escolar}\hspace{4.5cm}\textbf{Professor:} Nome do professor\\
		\textbf{Data:} 10/06/16\hspace{7.9cm}\textbf{Nota: }
		\begin{center}
			\hrulefill
		\end{center}
	\end{flushleft}

	\section*{\small{O tempo de realização da prova é de 10:30 às 12:10. }}

	\begin{enumerate}
		\item \textsf{(2,0 pontos)} Sendo as funções $f(1) = u, f(2) = v, f(3) = w, f(4) = x$. Formalmente, $G = (V_g,E_g)$ e $H = (V_h,E_h)$ são dito etc, então descreva.\\[0.3cm]
		\textbf{R.:}\\[0.6cm]

		\item \textsf{(1,5 pontos)} Descreva o problema de grafos bicoloráveis e informe em que classe de complexidade está esse problema.\\[0.3cm]
		\textbf{R.:}\\[0.6cm]

		\item \textsf{(2,5 pontos)} You are given a rectangular board of $M \times N$. Also you are given an unlimited number of standard domino pieces of $2 \times 1$. 

		\begin{itemize}
			\item Each domino completely covers two squares;
			\item No two dominoes overlap;
		\end{itemize}

		\textbf{R.:}\\[1cm]		
		\begin{flushright}
			\textit{Boa sorte!}
		\end{flushright}
	\end{enumerate}
\end{document}